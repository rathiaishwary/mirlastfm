\section{Auswertung der Daten}


\subsection{Datenquellen}
\subsubsection{Textbasiertes Genre-Artist-Mapping}
Verwendet f�r...

\subsubsection{LastFM}
Verwendet f�r Albeninfos f�r Epochenberechnung und f�r TagClouds, auch als Ground-Truth f�r die Klassifizierung.

\subsubsection{Google}
TagClouds durch Termfrequency

\subsection{Feature Extraction}
\subsubsection{Albumbasierte Epoche}

\subsubsection{TagClouds}


\subsection{�hnlichkeitsmessungen}
\subsubsection{Differenz von skalaren Werten}


\subsubsection{Cosinus-Similarity}
Ein Ma"s zur Bestimmung der "Ahnlichkeit der K"unstler in der LastFM Datenbank ist die Auswertung der Tags und ein Vergleich auf deren "Ubereinstimmung.
Dazu wurden die Tags der K"unstler mit der API ausgelesen und anschlie"send die "Ahnlichkeit mit der aus der LV bekannten Formel der Cosine Similarity bestimmt:

\begin{equation}
sim(a,b) = \frac{a*b}{\left|a\right|*\left|b\right|}
\end{equation}

Die API gibt auf Anfrage die Tags eines K"unstlers/Band mit einer Gewichtung zwischen 100 und 1 zur"uck. Zwei kleine Probleme traten hier aber auf: Zum werden auf jeden Fall eine Menge Tags zur"uckgegeben. Nach ein paar Versuchen wurde aber klar, dass die Tags mit Gewicht 1 nur zuf"alliger Natur sind und in die Berechnung nicht mit einzubeziehen sind (Mozart hat die Attribute \textit{psychodelic} bzw. \textit{Heavy-Metal} eher nicht verdient). Weiters lieferte unsere Wrapper-API f"ur Java leider nur die Tags, aber ohne Gewichtung. Die API musste daher geringf"ugig modifiziert werden.

\subsection{Visualisierung}

