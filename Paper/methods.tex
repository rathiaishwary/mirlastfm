\section{Auswertung der Daten}


\subsection{Datenquellen}
\subsubsection{Textbasiertes Genre-Artist-Mapping}
Verwendet f�r...

\subsubsection{LastFM}
Verwendet f�r Albeninfos f�r Berechnung der Wirkungszeit und f�r TagClouds, auch als Ground-Truth f�r die Klassifizierung, sowie f�r die Berechnung der aktivsten Wirkungszeit eines K�nstlers basierend auf den Releasedates seiner Alben. Dazu braucht man einen aktiven Account und api-key,.... (JAKOB)


\subsubsection{Google}
TagClouds durch Termfrequency

\subsection{Feature Extraction}
\subsubsection{Albumbasierte Wirkungszeit} \label{feature_wirkungszeit}
F�r die aktivste Wirkungszeit eines K�nstlers wurde ein Feature (skalarer Datenwert) berechnet, das mit Hilfe der LastFM-API ermittelt wurde. Die Wirkungszeit eines K�nstlers kann (muss aber nicht) repr�sentativ f�r eine bestimmte Epoche sein. Der letzte Fall trifft vor allem bei K�nstlern zu, die �ber mehrere Jahrzehnte hinweg aktiv sind, da sie nicht zwingend einer herk�mmlichen Epoche wie den 70ern oder 80ern, etc.\ zugeordnet werden k�nnen. 

Die aktivste Wirkungszeit wurde anhand der Anzahl und des Erscheinungsjahres der ver�ffentlichten Alben eines K�nstlers berechnet. Dazu wurden s�mtliche Alben inkl. deren Releasedates mit Hilfe der LastFM-API extrahiert. Die API stellt dazu die Funktion \texttt{artist.getTopAlbums} zur Verf�gung. Von den zur�ckgegebenen Alben k�nnen in der Folge mit der Funktion \texttt{album.getInfo} n�here Infos, u.a. das ReleaseDate des Albums, abgefragt werden.

F�r die Berechnung des Features wurden s�mtliche Releasedates addiert und ihr \textit{Mittelwert} gebildet. Es soll das mittlere Jahr der gesamten (bisherigen) Wirkungszeit repr�sentieren. 

\subsubsection{TagClouds}


\subsection{�hnlichkeitsmessungen}
\subsubsection{Differenz von skalaren Werten}
Das in Kap.\ \ref{feature_wirkungszeit} berechnete Feature wurde f�r die Berechnung des �hnlichkeitsma�es verwendet. Dieses soll darstellen, welche K�nstler zur selben Zeit am aktivsten (hinsichtlicher der produzierten Alben) waren. Optimalerweise l�sst sich daraus auch eine gewisse Genre�hnlichkeit der K�nstler feststellen, da gewisse Genres, z.B. Electronic, Punk, etc. sehr typisch f�r gewisse Zeitepochen der Geschichte sind. F�r Rock, Pop oder Heavy Metall trifft das kaum oder gar nicht zu.

Die H�he der �hnlichkeit wurde aus der Distanz zwischen den skalaren Featurewerten zweier K�nstler berechnet. Je n�her sich die Features selbst waren (�hnliche Jahreszahlen), desto geringer ist die Differenz. Folglich musste diese Distanz (nach erfolgter Normalisierung anhand der Featurewerte) invertiert werden, damit h�here Werte ein h�heres Ma� an �hnlichkeit ausdr�cken. Letztendlich bewegt sich diese zwischen den Werten 0.0 (keine �hnlichkeit) und 1.0 (idente aktivste Wirkungszeit). 

\subsubsection{Cosinus-Similarity}
Ein Ma"s zur Bestimmung der "Ahnlichkeit der K"unstler in der LastFM Datenbank ist die Auswertung der Tags und ein Vergleich auf deren "Ubereinstimmung.
Dazu wurden die Tags der K"unstler mit der API ausgelesen und anschlie"send die "Ahnlichkeit mit der aus der LV bekannten Formel der Cosine Similarity bestimmt:

\begin{equation}
sim(a,b) = \frac{a*b}{\left|a\right|*\left|b\right|}
\end{equation}

Die API gibt auf Anfrage die Tags eines K"unstlers/Band mit einer Gewichtung zwischen 100 und 1 zur"uck. Zwei kleine Probleme traten hier aber auf: Zum werden auf jeden Fall eine Menge Tags zur"uckgegeben. Nach ein paar Versuchen wurde aber klar, dass die Tags mit Gewicht 1 nur zuf"alliger Natur sind und in die Berechnung nicht mit einzubeziehen sind (Mozart hat die Attribute \textit{psychodelic} bzw. \textit{Heavy-Metal} eher nicht verdient). Weiters lieferte unsere Wrapper-API f"ur Java leider nur die Tags, aber ohne Gewichtung. Die API musste daher geringf"ugig modifiziert werden.

\subsection{Visualisierung}

