\section{LastFM}
\subsection{Hintergrund}
\textit{LastFM} ist heute ein personalisiertes Online-Radio - abspielbar mit einer eigenen, propri"ateren Software. Dieses erm"ogicht es den Usern einen auf ihren Geschmack zugeschnittenen Musikstream abzuspielen. Angegeben kann der momentan gew"unschte Musikstil via Tags - entweder auf einen K"unstler bezogen oder auf einen Musikstil bzw. Genre. Ebenso ist es m"oglich "uber Plug-Ins das Abspielverhalten von Usern in anderen Audio-Playern zu beobachten um damit die Erstellung des Nutzerprofils zu unterst"utzen.
Weiters ist \textit{LastFM} auch ein gro"ses soziales Netzwerk, dass vor allem auf das zusammenbringen von Menschen mit "ahnlichem Musikgeschmack ausgelegt ist. Die User der Community tragen zur Klassifikation des Musikbestandes durch tagging, Wiki-Betr"age und einfach ihr H"orverhalten bei. Mehr dazu in Abschnitt \ref{features_lastfm}.

\subsection{Geschichtlicher "Uberblick}
\textit{LastFM} wurde in den sp"aten 90er Jahren als Online-Musiklabel gegr"undet und bot als Feature die M"oglichkeit, sich durch die Art der konsumierten Musikst"ucke ein Profil "uber seinen Musikgeschmack zu erstellen. Die Firma \textit{Audioscrobbler}, hervorgegangen aus einem Informatikprojekt, hatte sehr "ahnliche Ideen, worauf hin beide Unternehmen sehr eng zusammenarbeiteten, bis sie schlie"slich im Jahr 2005 fusionierten und unter dem Namen \textit{LastFM} die Funktionen von beiden Technologien zur Verf"ugung stellen.
2007 wurde \textit{LastFM} um den Preis von 280 Millionen Dollar an das US-amerikanische Medienunternehmen CBS verkauft. Diese "Ubernahme geh"ort damit zu den gr"o"sten dotcom Aquisitionen bisher. 
Seit April 2009 ist die Benutzung des Radiodienstes nur mehr in den USA, Gro"sbritannien und Deutschland kostenlos m"oglich. In anderen L"andern muss ein Entgelt von 3 Euro f"ur die monatliche Nutzung erbracht werden.



